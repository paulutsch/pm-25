
\documentclass{article}
\usepackage{colortbl}
\usepackage{makecell}
\usepackage{multirow}
\usepackage{supertabular}

\begin{document}

\newcounter{utterance}

\twocolumn

{ \footnotesize  \setcounter{utterance}{1}
\setlength{\tabcolsep}{0pt}
\begin{supertabular}{c@{$\;$}|p{.15\linewidth}@{}p{.15\linewidth}p{.15\linewidth}p{.15\linewidth}p{.15\linewidth}p{.15\linewidth}}

    \# & $\;$A & \multicolumn{4}{c}{Game Master} & $\;\:$B\\
    \hline 

    \theutterance \stepcounter{utterance}  

    & & \multicolumn{4}{p{0.6\linewidth}}{\cellcolor[rgb]{0.9,0.9,0.9}{%
	\makecell[{{p{\linewidth}}}]{% 
	  \tt {\tiny [A$\langle$GM]}  
	 Let us do the following task. The goal is to fill an empty grid that looks like this:\\ \tt \\ \tt ▢ ▢ ▢ ▢ ▢\\ \tt ▢ ▢ ▢ ▢ ▢\\ \tt ▢ ▢ ▢ ▢ ▢\\ \tt ▢ ▢ ▢ ▢ ▢\\ \tt ▢ ▢ ▢ ▢ ▢\\ \tt \\ \tt A filled grid below is 5 by 5 and can look like this:\\ \tt \\ \tt ▢ ▢ ▢ ▢ ▢\\ \tt ▢ ▢ E ▢ ▢\\ \tt ▢ ▢ ▢ ▢ ▢\\ \tt ▢ ▢ ▢ ▢ ▢\\ \tt X X X X X\\ \tt \\ \tt I want you to describe this grid to me, step by step. You don't need to describe the empty squares, which are denoted with "▢". Only describe the location of letters in the grid. Then you wait for me to say "What is your next command?", and then you continue with the next step. Take the size of the grid into consideration while giving commands. You can give only one command at each turn but it can refer to single or multiple cells, rows, columns or even the whole grid at once. When you have described everything, you say "Command: DONE". Only give the command and omit everything else.\\ \tt \\ \tt For the filled grid above, here are the example steps.\\ \tt \\ \tt Command: Put an E in second row third column\\ \tt \\ \tt What is your next command?\\ \tt Command: Fill the last row with X\\ \tt \\ \tt What is your next command?\\ \tt Command: DONE\\ \tt \\ \tt Ok. Please do this for the following example, which is a 5 by 5 grid.\\ \tt Z ▢ ▢ ▢ Z\\ \tt Z Z ▢ Z Z\\ \tt Z ▢ Z ▢ Z\\ \tt Z ▢ ▢ ▢ Z\\ \tt Z ▢ ▢ ▢ Z\\ \tt What is your next command? 
	  } 
	   } 
	   } 
	 & \\ 
 

    \theutterance \stepcounter{utterance}  

    & \multicolumn{4}{p{0.6\linewidth}}{\cellcolor[rgb]{0.8,1,0.9}{%
	\makecell[{{p{\linewidth}}}]{% 
	  \tt {\tiny [A$\rangle$GM]}  
	 The user wants to fill their 5x5 grid step by step. They provided a similar example where an E was placed in a specific cell and then the last row was filled with Xs.\\ \tt \\ \tt Given the new grid:\\ \tt \\ \tt Z ▢ ▢ ▢ Z\\ \tt Z Z ▢ Z Z\\ \tt Z ▢ Z ▢ Z\\ \tt Z ▢ ▢ ▢ Z\\ \tt Z ▢ ▢ ▢ Z\\ \tt \\ \tt The task is to determine where to place Xs. Following the example, I should place an X in a specific cell and then fill the fifth row.\\ \tt \\ \tt First, let's place an X in the center of the fifth row.\\ \tt \\ \tt Command: Put an X in fifth row third column.\\ \tt \\ \tt Then, fill the rest of the fifth row with Xs.\\ \tt \\ \tt But since I need to give only one command, perhaps it's better to combine them into filling columns 2-4 of the fifth row with Xs.\\ \tt \\ \tt Alternatively, follow the example by placing Xs in the entire last row, including any existing Zs. But that might not be intended. Therefore, focusing on the empty cells:\\ \tt \\ \tt Command: Fill fifth row from column 2 to 4 with X.\\ \tt \\ \tt But perhaps a clearer way is to specify each position.\\ \tt \\ \tt However, following the initial approach, here's the step-by-step thought process, ending with the final command.\\ \tt \\ \tt Final Command: Fill the fifth row with X in columns 2-4. But since we can only describe the location of letters, not the empty cells, maybe it's better phrased as filling columns 2-4.\\ \tt \\ \tt But to avoid ambiguity and stick to placing letters, I think the correct approach is:\\ \tt \\ \tt Command: Put an X in fifth row second, third, and fourth columns.\\ \tt \\ \tt So, the final command would be:\\ \tt \\ \tt Command: Put an X in fifth row second, third, and fourth columns.\\ \tt \\ \tt Alternatively, if a single command can fill multiple cells:\\ \tt \\ \tt Command: Fill the fifth row with X from column 2 to 4.\\ \tt \\ \tt But since they specified to only describe letter locations without mentioning empty squares, perhaps it's better to list each position. However, in the example, they filled an entire row, so maybe that's acceptable.\\ \tt \\ \tt Given the above, the final command is:\\ \tt \\ \tt Command: Fill fifth row with X.\\ \tt \\ \tt But to be precise and avoid overwriting Zs, perhaps it's:\\ \tt \\ \tt Command: Put an X in fifth row columns 2,3,4.\\ \tt \\ \tt So, I think that's the correct approach.\\ \tt \\ \tt Final Command: Put an X in fifth row columns 2,3,4.\\ \tt \\ \tt But considering the example filled the entire last row with X, including the cell where E was placed, perhaps it's acceptable to overwrite.\\ \tt \\ \tt Thus, final command:\\ \tt \\ \tt Command: Fill the fifth row with X. 
	  } 
	   } 
	   } 
	 & & \\ 
 

    \theutterance \stepcounter{utterance}  

    & & & \multicolumn{2}{p{0.3\linewidth}}{\cellcolor[rgb]{0.95,0.95,0.95}{%
	\makecell[{{p{\linewidth}}}]{% 
	  \tt {\tiny [GM$|$GM]}  
	 Invalid instruction format 
	  } 
	   } 
	   } 
	 & & \\ 
 

\end{supertabular}
}

\end{document}
