
\documentclass{article}
\usepackage{colortbl}
\usepackage{makecell}
\usepackage{multirow}
\usepackage{supertabular}

\begin{document}

\newcounter{utterance}

\twocolumn

{ \footnotesize  \setcounter{utterance}{1}
\setlength{\tabcolsep}{0pt}
\begin{supertabular}{c@{$\;$}|p{.15\linewidth}@{}p{.15\linewidth}p{.15\linewidth}p{.15\linewidth}p{.15\linewidth}p{.15\linewidth}}

    \# & $\;$A & \multicolumn{4}{c}{Game Master} & $\;\:$B\\
    \hline 

    \theutterance \stepcounter{utterance}  

    & & \multicolumn{4}{p{0.6\linewidth}}{\cellcolor[rgb]{0.9,0.9,0.9}{%
	\makecell[{{p{\linewidth}}}]{% 
	  \tt {\tiny [A$\langle$GM]}  
	 You are participating in a collaborative guessing game.\\ \tt The goal is to find out whether this grid and another picture only I can see, are the same.\\ \tt Please describe your grid first. Then, I will provide my description and we can ask each other questions about the grids to figure out whether they are the same. Now start your short grid description with "DESCRIPTION:" followed by the description. Do not add anything else. This is your grid:\\ \tt X ▢ ▢ ▢ X\\ \tt ▢ X ▢ X ▢\\ \tt ▢ ▢ X ▢ ▢\\ \tt ▢ ▢ X ▢ ▢\\ \tt ▢ ▢ X ▢ ▢ 
	  } 
	   } 
	   } 
	 & \\ 
 

    \theutterance \stepcounter{utterance}  

    & \multicolumn{4}{p{0.6\linewidth}}{\cellcolor[rgb]{0.8,1,0.9}{%
	\makecell[{{p{\linewidth}}}]{% 
	  \tt {\tiny [A$\rangle$GM]}  
	 The described grid is a 5x5 arrangement featuring symmetry along the central vertical axis (Column3). The first row has Xs at both ends, while the second row follows a checkerboard pattern. From the third to fifth rows, each row starts and ends with diamonds, with an X in the middle. The columns show specific patterns where Column1 and Column5 have Xs only in certain rows, while Column3 has Xs from the third row onwards.\\ \tt \\ \tt The grid is symmetric along the central vertical axis (Column3), with a solid X column spanning Rows3 to 5. This unique combination of features helps identify it as the same grid as another if similar patterns are present.\\ \tt \\ \tt ---\\ \tt \\ \tt **Step-by-Step Explanation:**\\ \tt \\ \tt 1. **Grid Dimensions:** The described grid is 5 rows by 5 columns (5x5).\\ \tt \\ \tt 2. **Symmetry Check:** \\ \tt    - Reflecting the grid over Column3 reveals that each row mirrors itself across this central axis, indicating symmetry.\\ \tt    \\ \tt 3. **Row Patterns:**\\ \tt    - **Row1:** X at positions C1 and C5, with diamonds in between.\\ \tt    - **Row2:** Checkerboard pattern starting with a diamond, followed by X, then diamond, X, and ending with diamond.\\ \tt    - **Rows3-5:** Consistently have two diamonds, an X, and then two diamonds.\\ \tt \\ \tt 4. **Column Analysis:**\\ \tt    - Columns1 and5 have Xs only in specific rows (Row1 for Column1 and Row1 for Column5; Row2 for Column5).\\ \tt    - Column3 has Xs from Row3 to Row5.\\ \tt    \\ \tt 5. **Unique Features:** The combination of corner Xs, a checkerboard pattern, and a solid central column from Row3 onwards are key identifiers.\\ \tt \\ \tt If another description matches these structural elements, the grids are identical.\\ \tt \\ \tt ---\\ \tt \\ \tt **Answer:**\\ \tt Based on the described grid's symmetry, row patterns, and specific column features, it is indeed the same as another grid sharing these characteristics. 
	  } 
	   } 
	   } 
	 & & \\ 
 

    \theutterance \stepcounter{utterance}  

    & & & \multicolumn{2}{p{0.3\linewidth}}{\cellcolor[rgb]{0.95,0.95,0.95}{%
	\makecell[{{p{\linewidth}}}]{% 
	  \tt {\tiny [GM$|$GM]}  
	 abort, first word: The 
	  } 
	   } 
	   } 
	 & & \\ 
 

    \theutterance \stepcounter{utterance}  

    & & & \multicolumn{2}{p{0.3\linewidth}}{\cellcolor[rgb]{0.95,0.95,0.95}{%
	\makecell[{{p{\linewidth}}}]{% 
	  \tt {\tiny [GM$|$GM]}  
	 Aborted 
	  } 
	   } 
	   } 
	 & & \\ 
 

    \theutterance \stepcounter{utterance}  

    & & & \multicolumn{2}{p{0.3\linewidth}}{\cellcolor[rgb]{0.95,0.95,0.95}{%
	\makecell[{{p{\linewidth}}}]{% 
	  \tt {\tiny [GM$|$GM]}  
	 Aborted 
	  } 
	   } 
	   } 
	 & & \\ 
 

\end{supertabular}
}

\end{document}
