
\documentclass{article}
\usepackage{colortbl}
\usepackage{makecell}
\usepackage{multirow}
\usepackage{supertabular}

\begin{document}

\newcounter{utterance}

\twocolumn

{ \footnotesize  \setcounter{utterance}{1}
\setlength{\tabcolsep}{0pt}
\begin{supertabular}{c@{$\;$}|p{.15\linewidth}@{}p{.15\linewidth}p{.15\linewidth}p{.15\linewidth}p{.15\linewidth}p{.15\linewidth}}

    \# & $\;$A & \multicolumn{4}{c}{Game Master} & $\;\:$B\\
    \hline 

    \theutterance \stepcounter{utterance}  

    & & \multicolumn{4}{p{0.6\linewidth}}{\cellcolor[rgb]{0.9,0.9,0.9}{%
	\makecell[{{p{\linewidth}}}]{% 
	  \tt {\tiny [A$\langle$GM]}  
	 Let's play a guessing game. Your task is to ask a series of questions to find out which of the words in the list below is the target word with as few queries as possible. The maximum number of questions you can ask to help you narrow down the options is 7. Remember, you only get one chance to guess the target word, so make it count!\\ \tt \\ \tt The rules of the game are: \\ \tt \\ \tt 1. Only ask one question per turn. DO NOT ASK MORE THAN ONE QUESTION.\\ \tt \\ \tt 2. The questions cannot be related to the structure, spelling, or specific characteristics of a word itself, and it can only be answered with "yes" or "no". \\ \tt \\ \tt 3. Use the following format for asking questions and DO NOT ADD ANY TEXT OTHER THAN THIS: \\ \tt \\ \tt QUESTION: <your question>\\ \tt \\ \tt 4. You have only one opportunity to guess the target word. To make your guess, use the following format and DO NOT ADD ANY TEXT OTHER THAN THIS: \\ \tt \\ \tt GUESS: <target word>\\ \tt \\ \tt Remember you cannot ask for a hint. Make your question brief. Let’s start!\\ \tt \\ \tt List of candidate words: ['hexagon', 'nonagon', 'semicircle', 'oblong', 'evening', 'afternoon', 'nap', 'night'] 
	  } 
	   } 
	   } 
	 & \\ 
 

    \theutterance \stepcounter{utterance}  

    & \multicolumn{4}{p{0.6\linewidth}}{\cellcolor[rgb]{0.8,1,0.9}{%
	\makecell[{{p{\linewidth}}}]{% 
	  \tt {\tiny [A$\rangle$GM]}  
	 Alright, let's start with the first question to narrow down the possible candidates. \\ \tt \\ \tt 1. QUESTION: Is the target word one of the following: hexagon, nonagon, semicircle, or oblong?\\ \tt \\ \tt If your answer is "yes," proceed to the next step; if not, we'll focus on the remaining options.\\ \tt \\ \tt 2. QUESTION: Does the target word contain the letters "gon"? \\ \tt \\ \tt If the answer is "yes," narrow it down further with the next question. If not, determine whether it's a shape that ends with "circle."\\ \tt \\ \tt 3. QUESTION: Does the target word have the letter "a"?\\ \tt \\ \tt If the answer is "yes," guess "hexagon." If not, guess "nonagon."\\ \tt \\ \tt If you answered "no" to question 2:\\ \tt \\ \tt 4. QUESTION: Does the target word end with "circle"? \\ \tt \\ \tt If it does, guess "semicircle." Otherwise, guess "oblong."\\ \tt \\ \tt If your initial answer was "no":\\ \tt \\ \tt 5. QUESTION: Does the target word contain the letters "ap"?\\ \tt \\ \tt If yes, guess "nap." If not:\\ \tt \\ \tt 6. QUESTION: Does the target word end with "ing"?\\ \tt \\ \tt If yes, guess "evening." Otherwise, it must be either "afternoon" or "night." Given that we can't narrow this further without additional info, proceed to guess based on the most probable from the remaining options.\\ \tt \\ \tt 7. GUESS: night\\ \tt \\ \tt This approach ensures that each question reduces the possibilities as much as possible within the constraints.\\ \tt \\ \tt GUESS: night 
	  } 
	   } 
	   } 
	 & & \\ 
 

    \theutterance \stepcounter{utterance}  

    & & & \multicolumn{2}{p{0.3\linewidth}}{\cellcolor[rgb]{0.95,0.95,0.95}{%
	\makecell[{{p{\linewidth}}}]{% 
	  \tt {\tiny [GM$|$GM]}  
	 Invalid format. Guesser must use the form 'QUESTION: ' or 'GUESS: '. 
	  } 
	   } 
	   } 
	 & & \\ 
 

    \theutterance \stepcounter{utterance}  

    & & & \multicolumn{2}{p{0.3\linewidth}}{\cellcolor[rgb]{0.95,0.95,0.95}{%
	\makecell[{{p{\linewidth}}}]{% 
	  \tt {\tiny [GM$|$GM]}  
	 abort game 
	  } 
	   } 
	   } 
	 & & \\ 
 

\end{supertabular}
}

\end{document}
