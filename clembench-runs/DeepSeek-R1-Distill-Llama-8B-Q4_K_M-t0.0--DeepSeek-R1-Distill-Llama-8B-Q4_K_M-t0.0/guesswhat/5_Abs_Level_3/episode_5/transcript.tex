
\documentclass{article}
\usepackage{colortbl}
\usepackage{makecell}
\usepackage{multirow}
\usepackage{supertabular}

\begin{document}

\newcounter{utterance}

\twocolumn

{ \footnotesize  \setcounter{utterance}{1}
\setlength{\tabcolsep}{0pt}
\begin{supertabular}{c@{$\;$}|p{.15\linewidth}@{}p{.15\linewidth}p{.15\linewidth}p{.15\linewidth}p{.15\linewidth}p{.15\linewidth}}

    \# & $\;$A & \multicolumn{4}{c}{Game Master} & $\;\:$B\\
    \hline 

    \theutterance \stepcounter{utterance}  

    & & \multicolumn{4}{p{0.6\linewidth}}{\cellcolor[rgb]{0.9,0.9,0.9}{%
	\makecell[{{p{\linewidth}}}]{% 
	  \tt {\tiny [A$\langle$GM]}  
	 Let's play a guessing game. Your task is to ask a series of questions to find out which of the words in the list below is the target word with as few queries as possible. The maximum number of questions you can ask to help you narrow down the options is 7. Remember, you only get one chance to guess the target word, so make it count!\\ \tt \\ \tt The rules of the game are: \\ \tt \\ \tt 1. Only ask one question per turn. DO NOT ASK MORE THAN ONE QUESTION.\\ \tt \\ \tt 2. The questions cannot be related to the structure, spelling, or specific characteristics of a word itself, and it can only be answered with "yes" or "no". \\ \tt \\ \tt 3. Use the following format for asking questions and DO NOT ADD ANY TEXT OTHER THAN THIS: \\ \tt \\ \tt QUESTION: <your question>\\ \tt \\ \tt 4. You have only one opportunity to guess the target word. To make your guess, use the following format and DO NOT ADD ANY TEXT OTHER THAN THIS: \\ \tt \\ \tt GUESS: <target word>\\ \tt \\ \tt Remember you cannot ask for a hint. Make your question brief. Let’s start!\\ \tt \\ \tt List of candidate words: ['unicorn', 'monster', 'psychic', 'witch', 'vampire', 'goblin', 'angel', 'fairy'] 
	  } 
	   } 
	   } 
	 & \\ 
 

    \theutterance \stepcounter{utterance}  

    & \multicolumn{4}{p{0.6\linewidth}}{\cellcolor[rgb]{0.8,1,0.9}{%
	\makecell[{{p{\linewidth}}}]{% 
	  \tt {\tiny [A$\rangle$GM]}  
	 To determine the target word efficiently within the constraints of 7 questions, we will structure our questions to eliminate the most possibilities at each step. Here is the organized plan:\\ \tt \\ \tt 1. **Does the target word have a horn?**\\ \tt    - If yes: The word is "unicorn."\\ \tt    - If no: Proceed to question 2.\\ \tt \\ \tt 2. **If the answer was "no," does the target word feed on blood?**\\ \tt    - If yes: The word is "vampire."\\ \tt    - If no: Proceed to question 3.\\ \tt \\ \tt 3. **Is the target word related to magic or has supernatural abilities?**\\ \tt    - If yes: The word is either "witch," "psychic," or "fairy."\\ \tt    - If no: Proceed to question 4.\\ \tt \\ \tt 4. **If the answer was "no" in question 3, does the target word involve mischievous behavior?**\\ \tt    - If yes: The word is "goblin."\\ \tt    - If no: Proceed to question 5.\\ \tt \\ \tt 5. **Is the target word associated with being a guardian or spiritual figure with wings?**\\ \tt    - If yes: The word is "angel" or "fairy." \\ \tt    - If no: Proceed to question 6.\\ \tt \\ \tt 6. **If still unsure after question 5, ask if the target word is specifically a witch or psychic.**\\ \tt    - If yes: The word is either "witch" or "psychic."\\ \tt    - If no: The word is "fairy."\\ \tt \\ \tt 7. **Based on all previous answers, make your final guess.**\\ \tt \\ \tt After following this structured questioning approach, you can narrow down the target word effectively within 7 questions.\\ \tt \\ \tt GUESS: unicorn 
	  } 
	   } 
	   } 
	 & & \\ 
 

    \theutterance \stepcounter{utterance}  

    & & & \multicolumn{2}{p{0.3\linewidth}}{\cellcolor[rgb]{0.95,0.95,0.95}{%
	\makecell[{{p{\linewidth}}}]{% 
	  \tt {\tiny [GM$|$GM]}  
	 Invalid format. Guesser must use the form 'QUESTION: ' or 'GUESS: '. 
	  } 
	   } 
	   } 
	 & & \\ 
 

    \theutterance \stepcounter{utterance}  

    & & & \multicolumn{2}{p{0.3\linewidth}}{\cellcolor[rgb]{0.95,0.95,0.95}{%
	\makecell[{{p{\linewidth}}}]{% 
	  \tt {\tiny [GM$|$GM]}  
	 abort game 
	  } 
	   } 
	   } 
	 & & \\ 
 

\end{supertabular}
}

\end{document}
