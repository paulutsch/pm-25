
\documentclass{article}
\usepackage{colortbl}
\usepackage{makecell}
\usepackage{multirow}
\usepackage{supertabular}

\begin{document}

\newcounter{utterance}

\twocolumn

{ \footnotesize  \setcounter{utterance}{1}
\setlength{\tabcolsep}{0pt}
\begin{supertabular}{c@{$\;$}|p{.15\linewidth}@{}p{.15\linewidth}p{.15\linewidth}p{.15\linewidth}p{.15\linewidth}p{.15\linewidth}}

    \# & $\;$A & \multicolumn{4}{c}{Game Master} & $\;\:$B\\
    \hline 

    \theutterance \stepcounter{utterance}  

    & & \multicolumn{4}{p{0.6\linewidth}}{\cellcolor[rgb]{0.9,0.9,0.9}{%
	\makecell[{{p{\linewidth}}}]{% 
	  \tt {\tiny [A$\langle$GM]}  
	 Let's play a guessing game. Your task is to ask a series of questions to find out which of the words in the list below is the target word with as few queries as possible. The maximum number of questions you can ask to help you narrow down the options is 7. Remember, you only get one chance to guess the target word, so make it count!\\ \tt \\ \tt The rules of the game are: \\ \tt \\ \tt 1. Only ask one question per turn. DO NOT ASK MORE THAN ONE QUESTION.\\ \tt \\ \tt 2. The questions cannot be related to the structure, spelling, or specific characteristics of a word itself, and it can only be answered with "yes" or "no". \\ \tt \\ \tt 3. Use the following format for asking questions and DO NOT ADD ANY TEXT OTHER THAN THIS: \\ \tt \\ \tt QUESTION: <your question>\\ \tt \\ \tt 4. You have only one opportunity to guess the target word. To make your guess, use the following format and DO NOT ADD ANY TEXT OTHER THAN THIS: \\ \tt \\ \tt GUESS: <target word>\\ \tt \\ \tt Remember you cannot ask for a hint. Make your question brief. Let’s start!\\ \tt \\ \tt List of candidate words: ['Margarita', 'Martini', 'Liquor', 'Wine', 'Yard', 'Furlong', 'Centimeter', 'Kilometer'] 
	  } 
	   } 
	   } 
	 & \\ 
 

    \theutterance \stepcounter{utterance}  

    & \multicolumn{4}{p{0.6\linewidth}}{\cellcolor[rgb]{0.8,1,0.9}{%
	\makecell[{{p{\linewidth}}}]{% 
	  \tt {\tiny [A$\rangle$GM]}  
	 After considering the structure of the word list and the constraints, here's a step-by-step plan to narrow down the target word effectively within seven queries.\\ \tt \\ \tt 1. **Question 1:** Does the target word have exactly four letters?\\ \tt    - This helps split the list into smaller subsets: Wine and Yard (4 letters), versus longer words.\\ \tt \\ \tt 2. **If Question 1 is "Yes":**\\ \tt    - **Question 2:** Does the target word contain the letter 'e'?\\ \tt      - If yes, it's Wine.\\ \tt      - If no, it's Yard.\\ \tt \\ \tt 3. **If Question 1 is "No":** Target word has 6-10 letters.\\ \tt    - **Question 2:** Does the target word start with 'C'? \\ \tt      - Yes: Centimeter\\ \tt      - No: Proceed to next question\\ \tt \\ \tt 4. **If Question 2 is "No" (not Centimeter):**\\ \tt    - **Question 3:** Does the target word start with 'L'?\\ \tt      - Yes: Liquor\\ \tt      - No: Continue\\ \tt \\ \tt 5. **If Question 3 is "No" (not Liquor):**\\ \tt    - **Question 4:** Does the target word start with 'F'? \\ \tt      - Yes: Furlong\\ \tt      - No: Proceed to next question\\ \tt \\ \tt 6. **If Question 4 is "No" (not Furlong):**\\ \tt    - The remaining options are Margarita and Martini.\\ \tt    - **Question 5:** Does the target word have more than seven letters?\\ \tt      - Yes: Margarita\\ \tt      - No: Martini\\ \tt \\ \tt 7. **Final Guess:** After narrowing down through these steps, based on the answers, guess the target word.\\ \tt \\ \tt This strategy efficiently narrows down the list with each question, ensuring that the target word is identified within seven queries.\\ \tt \\ \tt \\ \tt GUESS: Margarita 
	  } 
	   } 
	   } 
	 & & \\ 
 

    \theutterance \stepcounter{utterance}  

    & & & \multicolumn{2}{p{0.3\linewidth}}{\cellcolor[rgb]{0.95,0.95,0.95}{%
	\makecell[{{p{\linewidth}}}]{% 
	  \tt {\tiny [GM$|$GM]}  
	 Invalid format. Guesser must use the form 'QUESTION: ' or 'GUESS: '. 
	  } 
	   } 
	   } 
	 & & \\ 
 

    \theutterance \stepcounter{utterance}  

    & & & \multicolumn{2}{p{0.3\linewidth}}{\cellcolor[rgb]{0.95,0.95,0.95}{%
	\makecell[{{p{\linewidth}}}]{% 
	  \tt {\tiny [GM$|$GM]}  
	 abort game 
	  } 
	   } 
	   } 
	 & & \\ 
 

\end{supertabular}
}

\end{document}
