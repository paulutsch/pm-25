
\documentclass{article}
\usepackage{colortbl}
\usepackage{makecell}
\usepackage{multirow}
\usepackage{supertabular}

\begin{document}

\newcounter{utterance}

\twocolumn

{ \footnotesize  \setcounter{utterance}{1}
\setlength{\tabcolsep}{0pt}
\begin{supertabular}{c@{$\;$}|p{.15\linewidth}@{}p{.15\linewidth}p{.15\linewidth}p{.15\linewidth}p{.15\linewidth}p{.15\linewidth}}

    \# & $\;$A & \multicolumn{4}{c}{Game Master} & $\;\:$B\\
    \hline 

    \theutterance \stepcounter{utterance}  

    & & \multicolumn{4}{p{0.6\linewidth}}{\cellcolor[rgb]{0.9,0.9,0.9}{%
	\makecell[{{p{\linewidth}}}]{% 
	  \tt {\tiny [A$\langle$GM]}  
	 Let's play a guessing game. Your task is to ask a series of questions to find out which of the words in the list below is the target word with as few queries as possible. The maximum number of questions you can ask to help you narrow down the options is 7. Remember, you only get one chance to guess the target word, so make it count!\\ \tt \\ \tt The rules of the game are: \\ \tt \\ \tt 1. Only ask one question per turn. DO NOT ASK MORE THAN ONE QUESTION.\\ \tt \\ \tt 2. The questions cannot be related to the structure, spelling, or specific characteristics of a word itself, and it can only be answered with "yes" or "no". \\ \tt \\ \tt 3. Use the following format for asking questions and DO NOT ADD ANY TEXT OTHER THAN THIS: \\ \tt \\ \tt QUESTION: <your question>\\ \tt \\ \tt 4. You have only one opportunity to guess the target word. To make your guess, use the following format and DO NOT ADD ANY TEXT OTHER THAN THIS: \\ \tt \\ \tt GUESS: <target word>\\ \tt \\ \tt Remember you cannot ask for a hint. Make your question brief. Let’s start!\\ \tt \\ \tt List of candidate words: ['Raven', 'Crow', 'Starling', 'Oriole', 'Rugby', 'Soccer', 'Gymnastics', 'Racing'] 
	  } 
	   } 
	   } 
	 & \\ 
 

    \theutterance \stepcounter{utterance}  

    & \multicolumn{4}{p{0.6\linewidth}}{\cellcolor[rgb]{0.8,1,0.9}{%
	\makecell[{{p{\linewidth}}}]{% 
	  \tt {\tiny [A$\rangle$GM]}  
	 To efficiently narrow down the target word within 7 questions, follow this strategy:\\ \tt \\ \tt 1. **Identify if it's a bird:** Ask if the target is a bird. This splits the list into birds and sports.\\ \tt \\ \tt 2. **Check for 'i' in bird names:** For birds with "i," determine if they start with "O" (Oriole) or "S" (Starling).\\ \tt \\ \tt 3. **Determine remaining birds:** If not containing "i," identify if it starts with "R" (Raven) or "C" (Crow).\\ \tt \\ \tt For non-birds (sports):\\ \tt \\ \tt 1. **Check for 'e':** If the word contains "e," it's likely Soccer.\\ \tt \\ \tt 2. **Identify start letters:** If not, determine if it starts with "G" (Gymnastics), "R" (Racing or Rugby).\\ \tt \\ \tt 3. **Final sports check:** For remaining words, determine if it's Racing or Rugby.\\ \tt \\ \tt Proceeding step-by-step:\\ \tt \\ \tt 1. QUESTION: Is the target a bird?\\ \tt \\ \tt 2. QUESTION: Does the target contain the letter 'i'?\\ \tt \\ \tt 3. QUESTION: If it contains 'i', does it start with 'O'?\\ \tt \\ \tt 4. QUESTION: If not containing 'i', does it start with 'R'?\\ \tt \\ \tt 5. QUESTION: For non-birds, does the target contain the letter 'e'?\\ \tt \\ \tt 6. QUESTION: For non-'e' sports, does it start with 'G'?\\ \tt \\ \tt 7. QUESTION: For remaining non-starters, is it Racing or Rugby?\\ \tt \\ \tt Based on these responses, guess the word accordingly.\\ \tt \\ \tt **GUESS:** [Answer based on questions]\\ \tt \\ \tt ---\\ \tt \\ \tt ANSWER:\\ \tt \\ \tt GUESS: Oriole 
	  } 
	   } 
	   } 
	 & & \\ 
 

    \theutterance \stepcounter{utterance}  

    & & & \multicolumn{2}{p{0.3\linewidth}}{\cellcolor[rgb]{0.95,0.95,0.95}{%
	\makecell[{{p{\linewidth}}}]{% 
	  \tt {\tiny [GM$|$GM]}  
	 Invalid format. Guesser must use the form 'QUESTION: ' or 'GUESS: '. 
	  } 
	   } 
	   } 
	 & & \\ 
 

    \theutterance \stepcounter{utterance}  

    & & & \multicolumn{2}{p{0.3\linewidth}}{\cellcolor[rgb]{0.95,0.95,0.95}{%
	\makecell[{{p{\linewidth}}}]{% 
	  \tt {\tiny [GM$|$GM]}  
	 abort game 
	  } 
	   } 
	   } 
	 & & \\ 
 

\end{supertabular}
}

\end{document}
