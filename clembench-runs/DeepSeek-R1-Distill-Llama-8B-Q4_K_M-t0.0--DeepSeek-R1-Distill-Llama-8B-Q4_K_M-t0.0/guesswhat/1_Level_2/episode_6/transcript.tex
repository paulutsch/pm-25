
\documentclass{article}
\usepackage{colortbl}
\usepackage{makecell}
\usepackage{multirow}
\usepackage{supertabular}

\begin{document}

\newcounter{utterance}

\twocolumn

{ \footnotesize  \setcounter{utterance}{1}
\setlength{\tabcolsep}{0pt}
\begin{supertabular}{c@{$\;$}|p{.15\linewidth}@{}p{.15\linewidth}p{.15\linewidth}p{.15\linewidth}p{.15\linewidth}p{.15\linewidth}}

    \# & $\;$A & \multicolumn{4}{c}{Game Master} & $\;\:$B\\
    \hline 

    \theutterance \stepcounter{utterance}  

    & & \multicolumn{4}{p{0.6\linewidth}}{\cellcolor[rgb]{0.9,0.9,0.9}{%
	\makecell[{{p{\linewidth}}}]{% 
	  \tt {\tiny [A$\langle$GM]}  
	 Let's play a guessing game. Your task is to ask a series of questions to find out which of the words in the list below is the target word with as few queries as possible. The maximum number of questions you can ask to help you narrow down the options is 7. Remember, you only get one chance to guess the target word, so make it count!\\ \tt \\ \tt The rules of the game are: \\ \tt \\ \tt 1. Only ask one question per turn. DO NOT ASK MORE THAN ONE QUESTION.\\ \tt \\ \tt 2. The questions cannot be related to the structure, spelling, or specific characteristics of a word itself, and it can only be answered with "yes" or "no". \\ \tt \\ \tt 3. Use the following format for asking questions and DO NOT ADD ANY TEXT OTHER THAN THIS: \\ \tt \\ \tt QUESTION: <your question>\\ \tt \\ \tt 4. You have only one opportunity to guess the target word. To make your guess, use the following format and DO NOT ADD ANY TEXT OTHER THAN THIS: \\ \tt \\ \tt GUESS: <target word>\\ \tt \\ \tt Remember you cannot ask for a hint. Make your question brief. Let’s start!\\ \tt \\ \tt List of candidate words: ['Bicycle', 'Wagon', 'Motorcycle', 'Car', 'Battleship', 'Destroyer', 'Motorboat', 'Ferry'] 
	  } 
	   } 
	   } 
	 & \\ 
 

    \theutterance \stepcounter{utterance}  

    & \multicolumn{4}{p{0.6\linewidth}}{\cellcolor[rgb]{0.8,1,0.9}{%
	\makecell[{{p{\linewidth}}}]{% 
	  \tt {\tiny [A$\rangle$GM]}  
	 The strategy is to first split the list into two groups: vehicles and ships/boats. Then, within each group, use additional yes/no questions to narrow down further until only one word remains.\\ \tt \\ \tt 1. First Question: Is the target word a type of vehicle?  \\ \tt    - If yes, proceed with the vehicle group; else, proceed with the ship/boat group.\\ \tt \\ \tt 2a. Second Question (Vehicle group): Does the target end with 'cycle'?  \\ \tt    - Yes → Bicycle or Motorcycle. Ask if it contains 'motor' to decide.\\ \tt    - No → Wagon or Car. Determine if it's four-wheeled for Car or Wagon.\\ \tt \\ \tt 2b. Second Question (Ship/Boat group): Is it a military ship?  \\ \tt    - Yes → Battleship or Destroyer. Ask if starts with 'Bat' to choose.\\ \tt    - No → Motorboat or Ferry. Determine based on usage.\\ \tt \\ \tt 3a. Third Question (Vehicle, ends with cycle): Contains 'motor'?  \\ \tt    - Yes → Motorcycle; else, Bicycle.\\ \tt \\ \tt 3b. Third Question (Ship, military ship and starts with Bat): Starts with 'Bat'?  \\ \tt    - Yes → Battleship; else, Destroyer.\\ \tt \\ \tt 4a. Fourth Question (Vehicle group, not ending with cycle): Four-wheeled?  \\ \tt    - Yes → Car; else, Wagon.\\ \tt \\ \tt 4b. Fourth Question (Ship group, not military ship): Ferry or Motorboat.  \\ \tt    - Used for transporting people → Ferry; else, Motorboat.\\ \tt \\ \tt After this process, the target word can be identified within four questions.\\ \tt \\ \tt GUESS: <target_word> 
	  } 
	   } 
	   } 
	 & & \\ 
 

    \theutterance \stepcounter{utterance}  

    & & & \multicolumn{2}{p{0.3\linewidth}}{\cellcolor[rgb]{0.95,0.95,0.95}{%
	\makecell[{{p{\linewidth}}}]{% 
	  \tt {\tiny [GM$|$GM]}  
	 Invalid format. Guesser must use the form 'QUESTION: ' or 'GUESS: '. 
	  } 
	   } 
	   } 
	 & & \\ 
 

    \theutterance \stepcounter{utterance}  

    & & & \multicolumn{2}{p{0.3\linewidth}}{\cellcolor[rgb]{0.95,0.95,0.95}{%
	\makecell[{{p{\linewidth}}}]{% 
	  \tt {\tiny [GM$|$GM]}  
	 abort game 
	  } 
	   } 
	   } 
	 & & \\ 
 

\end{supertabular}
}

\end{document}
