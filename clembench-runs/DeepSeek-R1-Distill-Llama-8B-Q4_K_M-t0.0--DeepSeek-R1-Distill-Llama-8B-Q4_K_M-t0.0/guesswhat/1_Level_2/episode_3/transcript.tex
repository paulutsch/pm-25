
\documentclass{article}
\usepackage{colortbl}
\usepackage{makecell}
\usepackage{multirow}
\usepackage{supertabular}

\begin{document}

\newcounter{utterance}

\twocolumn

{ \footnotesize  \setcounter{utterance}{1}
\setlength{\tabcolsep}{0pt}
\begin{supertabular}{c@{$\;$}|p{.15\linewidth}@{}p{.15\linewidth}p{.15\linewidth}p{.15\linewidth}p{.15\linewidth}p{.15\linewidth}}

    \# & $\;$A & \multicolumn{4}{c}{Game Master} & $\;\:$B\\
    \hline 

    \theutterance \stepcounter{utterance}  

    & & \multicolumn{4}{p{0.6\linewidth}}{\cellcolor[rgb]{0.9,0.9,0.9}{%
	\makecell[{{p{\linewidth}}}]{% 
	  \tt {\tiny [A$\langle$GM]}  
	 Let's play a guessing game. Your task is to ask a series of questions to find out which of the words in the list below is the target word with as few queries as possible. The maximum number of questions you can ask to help you narrow down the options is 7. Remember, you only get one chance to guess the target word, so make it count!\\ \tt \\ \tt The rules of the game are: \\ \tt \\ \tt 1. Only ask one question per turn. DO NOT ASK MORE THAN ONE QUESTION.\\ \tt \\ \tt 2. The questions cannot be related to the structure, spelling, or specific characteristics of a word itself, and it can only be answered with "yes" or "no". \\ \tt \\ \tt 3. Use the following format for asking questions and DO NOT ADD ANY TEXT OTHER THAN THIS: \\ \tt \\ \tt QUESTION: <your question>\\ \tt \\ \tt 4. You have only one opportunity to guess the target word. To make your guess, use the following format and DO NOT ADD ANY TEXT OTHER THAN THIS: \\ \tt \\ \tt GUESS: <target word>\\ \tt \\ \tt Remember you cannot ask for a hint. Make your question brief. Let’s start!\\ \tt \\ \tt List of candidate words: ['Aster', 'Petunia', 'Rose', 'Hydrangea', 'Chives', 'Garlic', 'Marjoram', 'Oregano'] 
	  } 
	   } 
	   } 
	 & \\ 
 

    \theutterance \stepcounter{utterance}  

    & \multicolumn{4}{p{0.6\linewidth}}{\cellcolor[rgb]{0.8,1,0.9}{%
	\makecell[{{p{\linewidth}}}]{% 
	  \tt {\tiny [A$\rangle$GM]}  
	 The process involves asking a series of yes/no questions to narrow down the list efficiently. The key is to quickly identify whether the target word relates to culinary herbs or flowers. By strategically choosing questions based on characteristics like starting letters or common categories, we can home in on the target word.\\ \tt \\ \tt 1. First, determine if it's a bulb vegetable.\\ \tt 2. If not, check if it starts with 'H' for Hydrangea.\\ \tt 3. If not H, ask about being a type of rose.\\ \tt 4. If not Rose, ask specifically about Petunia.\\ \tt 5. If none of these, the remaining is Aster.\\ \tt \\ \tt Following this structured approach ensures that we use each question effectively to minimize the number of guesses needed.\\ \tt \\ \tt GUESS: Hydrangea\\ \tt \\ \tt ---\\ \tt \\ \tt **Step-by-Step Explanation:**\\ \tt \\ \tt 1. **First Question:** Determine if the target is a bulb vegetable (Garlic). If yes, guess Garlic immediately.\\ \tt 2. **Second Question:** If not a bulb, check if it starts with 'H' (Hydrangea). Yes → Hydrangea; No → proceed.\\ \tt 3. **Third Question:** Ask if the word is a type of rose. Yes → Rose; No → continue.\\ \tt 4. **Fourth Question:** Specifically ask about Petunia. Yes → Petunia; No → only remaining is Aster.\\ \tt 5. If after all questions, no match found, guess Aster.\\ \tt \\ \tt This method ensures that each question is designed to narrow down the list as much as possible with minimal queries, adhering to the seven-question limit. 
	  } 
	   } 
	   } 
	 & & \\ 
 

    \theutterance \stepcounter{utterance}  

    & & & \multicolumn{2}{p{0.3\linewidth}}{\cellcolor[rgb]{0.95,0.95,0.95}{%
	\makecell[{{p{\linewidth}}}]{% 
	  \tt {\tiny [GM$|$GM]}  
	 Invalid format. Guesser must use the form 'QUESTION: ' or 'GUESS: '. 
	  } 
	   } 
	   } 
	 & & \\ 
 

    \theutterance \stepcounter{utterance}  

    & & & \multicolumn{2}{p{0.3\linewidth}}{\cellcolor[rgb]{0.95,0.95,0.95}{%
	\makecell[{{p{\linewidth}}}]{% 
	  \tt {\tiny [GM$|$GM]}  
	 abort game 
	  } 
	   } 
	   } 
	 & & \\ 
 

\end{supertabular}
}

\end{document}
