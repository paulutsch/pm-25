
\documentclass{article}
\usepackage{colortbl}
\usepackage{makecell}
\usepackage{multirow}
\usepackage{supertabular}

\begin{document}

\newcounter{utterance}

\twocolumn

{ \footnotesize  \setcounter{utterance}{1}
\setlength{\tabcolsep}{0pt}
\begin{supertabular}{c@{$\;$}|p{.15\linewidth}@{}p{.15\linewidth}p{.15\linewidth}p{.15\linewidth}p{.15\linewidth}p{.15\linewidth}}

    \# & $\;$A & \multicolumn{4}{c}{Game Master} & $\;\:$B\\
    \hline 

    \theutterance \stepcounter{utterance}  

    & & \multicolumn{4}{p{0.6\linewidth}}{\cellcolor[rgb]{0.9,0.9,0.9}{%
	\makecell[{{p{\linewidth}}}]{% 
	  \tt {\tiny [A$\langle$GM]}  
	 Let's play a guessing game. Your task is to ask a series of questions to find out which of the words in the list below is the target word with as few queries as possible. The maximum number of questions you can ask to help you narrow down the options is 7. Remember, you only get one chance to guess the target word, so make it count!\\ \tt \\ \tt The rules of the game are: \\ \tt \\ \tt 1. Only ask one question per turn. DO NOT ASK MORE THAN ONE QUESTION.\\ \tt \\ \tt 2. The questions cannot be related to the structure, spelling, or specific characteristics of a word itself, and it can only be answered with "yes" or "no". \\ \tt \\ \tt 3. Use the following format for asking questions and DO NOT ADD ANY TEXT OTHER THAN THIS: \\ \tt \\ \tt QUESTION: <your question>\\ \tt \\ \tt 4. You have only one opportunity to guess the target word. To make your guess, use the following format and DO NOT ADD ANY TEXT OTHER THAN THIS: \\ \tt \\ \tt GUESS: <target word>\\ \tt \\ \tt Remember you cannot ask for a hint. Make your question brief. Let’s start!\\ \tt \\ \tt List of candidate words: ['Rattle', 'Thunder', 'Noisemaker', 'Fan', 'Kerosene', 'Gasoline', 'Tea', 'Milk'] 
	  } 
	   } 
	   } 
	 & \\ 
 

    \theutterance \stepcounter{utterance}  

    & \multicolumn{4}{p{0.6\linewidth}}{\cellcolor[rgb]{0.8,1,0.9}{%
	\makecell[{{p{\linewidth}}}]{% 
	  \tt {\tiny [A$\rangle$GM]}  
	 The process involves asking yes/no questions to narrow down the list of words by determining if they make noise or not, then further categorizing those that don't. The goal is to ask each question in a way that efficiently narrows down the possibilities while adhering to the constraints.\\ \tt \\ \tt 1. **Question 1:** Does the word make a noise?\\ \tt    - If yes: Proceed to category A (Rattle, Thunder, Noisemaker)\\ \tt    - If no: Proceed to category B (Fan, Kerosene, Gasoline, Tea, Milk)\\ \tt \\ \tt 2. **Question 2 (if in category B):** Does the word start with either "K" or "G"?\\ \tt    - If yes: Further narrow down to Kerosene/Gasoline\\ \tt    - If no: Consider if it's Fan, Tea, or Milk\\ \tt \\ \tt 3. **Question 3 (for category A):** Is the word "Rattle"? \\ \tt    - If yes: Target is Rattle\\ \tt    - If no: Determine if it starts with "T" for Thunder or not.\\ \tt \\ \tt 4. **Question 4:** For non-Fan, Kerosene/Gasoline cases:\\ \tt    - If in B and second question was no: Is the word Tea? Yes/No.\\ \tt \\ \tt This structured approach ensures that each target can be identified within the allowed number of questions, maximizing efficiency.\\ \tt \\ \tt Let's proceed with the questions.\\ \tt \\ \tt **QUESTION:** Does the target word make a noise?\\ \tt \\ \tt **QUESTION:** (If in category A) Is the target word "Rattle"?\\ \tt \\ \tt **QUESTION:** (If in category B and second question was yes) Is the target word "Kerosene"?\\ \tt \\ \tt ** QUESTION:** (If in category B and third question was no) Is the target word "Tea"? 
	  } 
	   } 
	   } 
	 & & \\ 
 

    \theutterance \stepcounter{utterance}  

    & & & \multicolumn{2}{p{0.3\linewidth}}{\cellcolor[rgb]{0.95,0.95,0.95}{%
	\makecell[{{p{\linewidth}}}]{% 
	  \tt {\tiny [GM$|$GM]}  
	 Invalid format. Guesser must use the form 'QUESTION: ' or 'GUESS: '. 
	  } 
	   } 
	   } 
	 & & \\ 
 

    \theutterance \stepcounter{utterance}  

    & & & \multicolumn{2}{p{0.3\linewidth}}{\cellcolor[rgb]{0.95,0.95,0.95}{%
	\makecell[{{p{\linewidth}}}]{% 
	  \tt {\tiny [GM$|$GM]}  
	 abort game 
	  } 
	   } 
	   } 
	 & & \\ 
 

\end{supertabular}
}

\end{document}
