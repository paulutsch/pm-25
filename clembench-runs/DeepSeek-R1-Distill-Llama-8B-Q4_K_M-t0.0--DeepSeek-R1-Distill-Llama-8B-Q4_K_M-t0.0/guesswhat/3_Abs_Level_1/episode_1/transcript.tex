
\documentclass{article}
\usepackage{colortbl}
\usepackage{makecell}
\usepackage{multirow}
\usepackage{supertabular}

\begin{document}

\newcounter{utterance}

\twocolumn

{ \footnotesize  \setcounter{utterance}{1}
\setlength{\tabcolsep}{0pt}
\begin{supertabular}{c@{$\;$}|p{.15\linewidth}@{}p{.15\linewidth}p{.15\linewidth}p{.15\linewidth}p{.15\linewidth}p{.15\linewidth}}

    \# & $\;$A & \multicolumn{4}{c}{Game Master} & $\;\:$B\\
    \hline 

    \theutterance \stepcounter{utterance}  

    & & \multicolumn{4}{p{0.6\linewidth}}{\cellcolor[rgb]{0.9,0.9,0.9}{%
	\makecell[{{p{\linewidth}}}]{% 
	  \tt {\tiny [A$\langle$GM]}  
	 Let's play a guessing game. Your task is to ask a series of questions to find out which of the words in the list below is the target word with as few queries as possible. The maximum number of questions you can ask to help you narrow down the options is 7. Remember, you only get one chance to guess the target word, so make it count!\\ \tt \\ \tt The rules of the game are: \\ \tt \\ \tt 1. Only ask one question per turn. DO NOT ASK MORE THAN ONE QUESTION.\\ \tt \\ \tt 2. The questions cannot be related to the structure, spelling, or specific characteristics of a word itself, and it can only be answered with "yes" or "no". \\ \tt \\ \tt 3. Use the following format for asking questions and DO NOT ADD ANY TEXT OTHER THAN THIS: \\ \tt \\ \tt QUESTION: <your question>\\ \tt \\ \tt 4. You have only one opportunity to guess the target word. To make your guess, use the following format and DO NOT ADD ANY TEXT OTHER THAN THIS: \\ \tt \\ \tt GUESS: <target word>\\ \tt \\ \tt Remember you cannot ask for a hint. Make your question brief. Let’s start!\\ \tt \\ \tt List of candidate words: ['stepbrother', 'mother', 'sample', 'quantify', 'disturbance', 'thrilled', 'dedication', 'gentle'] 
	  } 
	   } 
	   } 
	 & \\ 
 

    \theutterance \stepcounter{utterance}  

    & \multicolumn{4}{p{0.6\linewidth}}{\cellcolor[rgb]{0.8,1,0.9}{%
	\makecell[{{p{\linewidth}}}]{% 
	  \tt {\tiny [A$\rangle$GM]}  
	 To efficiently narrow down the target word within the constraints, I will ask questions that eliminate as many options as possible. Here's the optimal sequence:\\ \tt \\ \tt 1. QUESTION: Does the target word contain "brother"?\\ \tt \\ \tt    This helps identify 'stepbrother' if present.\\ \tt \\ \tt 2. If yes: GUESS stepbrother.\\ \tt \\ \tt    If no:\\ \tt \\ \tt       2. QUESTION: Does the target word start with "m" (i.e., mother)?\\ \tt \\ \tt          Identifies 'mother'.\\ \tt \\ \tt 3. If yes: GUESS mother.\\ \tt \\ \tt          If no:\\ \tt \\ \tt              Proceed to next attributes for other words.\\ \tt \\ \tt 4. QUESTION: Does the target word contain "sam"? Identifies 'sample' if present.\\ \tt \\ \tt 5. If yes: GUESS sample.\\ \tt \\ \tt 6. Else: QUESTION: Does the target word end with "ance"? Identifies 'disturbance'.\\ \tt \\ \tt 7. If yes: GUESS disturbance.\\ \tt \\ \tt 8. Else: QUESTION: Does the target word start with "de"? Identifies 'dedication' if present.\\ \tt \\ \tt 9. If yes: GUESS dedication.\\ \tt \\ \tt 10. Else: QUESTION: Does the target word end with "tion"? Identifies 'dedication'.\\ \tt \\ \tt 11. If yes: GUESS dedication.\\ \tt \\ \tt 12. Else: Make an educated guess among remaining words (quantify, thrilled, gentle). Based on commonality, guess 'gentle' as it is a commonly used positive adjective.\\ \tt \\ \tt However, since only seven questions are allowed, the sequence would be adjusted to fit within that limit:\\ \tt \\ \tt 1. QUESTION: Does the target word contain "brother"?\\ \tt \\ \tt    If yes: GUESS stepbrother.\\ \tt \\ \tt 2. Else, does it start with 'm'? GUESS mother if yes.\\ \tt \\ \tt 3. Else, contains "sam"? GUESS sample.\\ \tt \\ \tt 4. Ends with 'ance'? GUESS disturbance.\\ \tt \\ \tt 5. Starts with 'de'? GUESS dedication.\\ \tt \\ \tt 6. Ends with "tion"? GUESS dedication.\\ \tt \\ \tt 7. If none match, make an educated guess based on remaining options (quantify, thrilled, gentle). Perhaps choose the most likely adjective if no prior matches, which would be 'gentle'.\\ \tt \\ \tt **Final Answer:**\\ \tt \\ \tt GUESS: gentle 
	  } 
	   } 
	   } 
	 & & \\ 
 

    \theutterance \stepcounter{utterance}  

    & & & \multicolumn{2}{p{0.3\linewidth}}{\cellcolor[rgb]{0.95,0.95,0.95}{%
	\makecell[{{p{\linewidth}}}]{% 
	  \tt {\tiny [GM$|$GM]}  
	 Invalid format. Guesser must use the form 'QUESTION: ' or 'GUESS: '. 
	  } 
	   } 
	   } 
	 & & \\ 
 

    \theutterance \stepcounter{utterance}  

    & & & \multicolumn{2}{p{0.3\linewidth}}{\cellcolor[rgb]{0.95,0.95,0.95}{%
	\makecell[{{p{\linewidth}}}]{% 
	  \tt {\tiny [GM$|$GM]}  
	 abort game 
	  } 
	   } 
	   } 
	 & & \\ 
 

\end{supertabular}
}

\end{document}
