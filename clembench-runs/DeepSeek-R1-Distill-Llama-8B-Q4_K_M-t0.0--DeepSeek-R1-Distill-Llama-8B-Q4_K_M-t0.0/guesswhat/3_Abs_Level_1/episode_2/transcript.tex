
\documentclass{article}
\usepackage{colortbl}
\usepackage{makecell}
\usepackage{multirow}
\usepackage{supertabular}

\begin{document}

\newcounter{utterance}

\twocolumn

{ \footnotesize  \setcounter{utterance}{1}
\setlength{\tabcolsep}{0pt}
\begin{supertabular}{c@{$\;$}|p{.15\linewidth}@{}p{.15\linewidth}p{.15\linewidth}p{.15\linewidth}p{.15\linewidth}p{.15\linewidth}}

    \# & $\;$A & \multicolumn{4}{c}{Game Master} & $\;\:$B\\
    \hline 

    \theutterance \stepcounter{utterance}  

    & & \multicolumn{4}{p{0.6\linewidth}}{\cellcolor[rgb]{0.9,0.9,0.9}{%
	\makecell[{{p{\linewidth}}}]{% 
	  \tt {\tiny [A$\langle$GM]}  
	 Let's play a guessing game. Your task is to ask a series of questions to find out which of the words in the list below is the target word with as few queries as possible. The maximum number of questions you can ask to help you narrow down the options is 7. Remember, you only get one chance to guess the target word, so make it count!\\ \tt \\ \tt The rules of the game are: \\ \tt \\ \tt 1. Only ask one question per turn. DO NOT ASK MORE THAN ONE QUESTION.\\ \tt \\ \tt 2. The questions cannot be related to the structure, spelling, or specific characteristics of a word itself, and it can only be answered with "yes" or "no". \\ \tt \\ \tt 3. Use the following format for asking questions and DO NOT ADD ANY TEXT OTHER THAN THIS: \\ \tt \\ \tt QUESTION: <your question>\\ \tt \\ \tt 4. You have only one opportunity to guess the target word. To make your guess, use the following format and DO NOT ADD ANY TEXT OTHER THAN THIS: \\ \tt \\ \tt GUESS: <target word>\\ \tt \\ \tt Remember you cannot ask for a hint. Make your question brief. Let’s start!\\ \tt \\ \tt List of candidate words: ['uncle', 'grandparent', 'funeral', 'birthday', 'cell', 'oceanography', 'sulky', 'intimidation'] 
	  } 
	   } 
	   } 
	 & \\ 
 

    \theutterance \stepcounter{utterance}  

    & \multicolumn{4}{p{0.6\linewidth}}{\cellcolor[rgb]{0.8,1,0.9}{%
	\makecell[{{p{\linewidth}}}]{% 
	  \tt {\tiny [A$\rangle$GM]}  
	 To efficiently narrow down the target word from the list, here's a structured approach using the allowed number of questions (up to 7) and ensuring each question is informative:\\ \tt \\ \tt 1. **First Question:** Does the target word have any connection to family relationships? (uncle or grandparent)\\ \tt    - If yes: Target is either 'uncle' or 'grandparent'.\\ \tt    - If no: Eliminate both.\\ \tt \\ \tt 2. **Second Question (if needed):** If not related to family, does the target word relate to scientific fields? (cell or oceanography)\\ \tt    - If yes: Narrow down to 'cell' or 'oceanography'.\\ \tt    - If no: Eliminate both.\\ \tt \\ \tt 3. **Third Question:** Does the target word describe a specific emotion or feeling? (sulky)\\ \tt    - If yes: Target is 'sulky'.\\ \tt    - If no: Eliminate.\\ \tt \\ \tt 4. **Fourth Question:** If still in the running, does the target word relate to transportation? (sulky relates to horse carriages)\\ \tt    - If yes: Target is 'sulky'.\\ \tt    - Else if related to biology: Target is 'cell'.\\ \tt \\ \tt 5. **Fifth Question (if needed):** If not previously identified, consider if the target word relates to specific events like birthdays or funerals.\\ \tt    - If 'birthday' fits: guess accordingly.\\ \tt    - Else: narrow down further.\\ \tt \\ \tt 6. **Sixth Question:** If still unsure, consider if the word ends with specific letters or characteristics that help eliminate options.\\ \tt \\ \tt 7. **Final Guess:** Based on accumulated information, make an educated guess from the narrowed-down list.\\ \tt \\ \tt However, considering time constraints and optimal efficiency, here's a concise approach within 7 questions:\\ \tt \\ \tt 1. Does the target word start with 's'? (sulky)\\ \tt    - Yes: Sulky\\ \tt    - No: Eliminate\\ \tt \\ \tt 2. If not 's', does it relate to family? (uncle or grandparent)\\ \tt    - Yes: Uncle/Grandparent\\ \tt    - No: Eliminate\\ \tt \\ \tt 3. Does the target word have fewer than 6 letters? (cell/sulky)\\ \tt    - Yes: Narrow down further\\ \tt    - No: Eliminate cell and sulky\\ \tt \\ \tt 4. If still in running, does it relate to celebrations? (birthday)\\ \tt    - Yes: Birthday\\ \tt    - No: Eliminate\\ \tt \\ \tt 5. If not birthday, check for specific industry terms.\\ \tt \\ \tt 6. Final guess based on accumulated info.\\ \tt \\ \tt Given the constraints, here's the optimized path:\\ \tt \\ \tt GUESS: birthday 
	  } 
	   } 
	   } 
	 & & \\ 
 

    \theutterance \stepcounter{utterance}  

    & & & \multicolumn{2}{p{0.3\linewidth}}{\cellcolor[rgb]{0.95,0.95,0.95}{%
	\makecell[{{p{\linewidth}}}]{% 
	  \tt {\tiny [GM$|$GM]}  
	 Invalid format. Guesser must use the form 'QUESTION: ' or 'GUESS: '. 
	  } 
	   } 
	   } 
	 & & \\ 
 

    \theutterance \stepcounter{utterance}  

    & & & \multicolumn{2}{p{0.3\linewidth}}{\cellcolor[rgb]{0.95,0.95,0.95}{%
	\makecell[{{p{\linewidth}}}]{% 
	  \tt {\tiny [GM$|$GM]}  
	 abort game 
	  } 
	   } 
	   } 
	 & & \\ 
 

\end{supertabular}
}

\end{document}
