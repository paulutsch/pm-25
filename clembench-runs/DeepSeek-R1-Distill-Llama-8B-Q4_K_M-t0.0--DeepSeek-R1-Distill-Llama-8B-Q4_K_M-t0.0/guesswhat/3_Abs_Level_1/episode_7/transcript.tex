
\documentclass{article}
\usepackage{colortbl}
\usepackage{makecell}
\usepackage{multirow}
\usepackage{supertabular}

\begin{document}

\newcounter{utterance}

\twocolumn

{ \footnotesize  \setcounter{utterance}{1}
\setlength{\tabcolsep}{0pt}
\begin{supertabular}{c@{$\;$}|p{.15\linewidth}@{}p{.15\linewidth}p{.15\linewidth}p{.15\linewidth}p{.15\linewidth}p{.15\linewidth}}

    \# & $\;$A & \multicolumn{4}{c}{Game Master} & $\;\:$B\\
    \hline 

    \theutterance \stepcounter{utterance}  

    & & \multicolumn{4}{p{0.6\linewidth}}{\cellcolor[rgb]{0.9,0.9,0.9}{%
	\makecell[{{p{\linewidth}}}]{% 
	  \tt {\tiny [A$\langle$GM]}  
	 Let's play a guessing game. Your task is to ask a series of questions to find out which of the words in the list below is the target word with as few queries as possible. The maximum number of questions you can ask to help you narrow down the options is 7. Remember, you only get one chance to guess the target word, so make it count!\\ \tt \\ \tt The rules of the game are: \\ \tt \\ \tt 1. Only ask one question per turn. DO NOT ASK MORE THAN ONE QUESTION.\\ \tt \\ \tt 2. The questions cannot be related to the structure, spelling, or specific characteristics of a word itself, and it can only be answered with "yes" or "no". \\ \tt \\ \tt 3. Use the following format for asking questions and DO NOT ADD ANY TEXT OTHER THAN THIS: \\ \tt \\ \tt QUESTION: <your question>\\ \tt \\ \tt 4. You have only one opportunity to guess the target word. To make your guess, use the following format and DO NOT ADD ANY TEXT OTHER THAN THIS: \\ \tt \\ \tt GUESS: <target word>\\ \tt \\ \tt Remember you cannot ask for a hint. Make your question brief. Let’s start!\\ \tt \\ \tt List of candidate words: ['postman', 'policeman', 'count', 'marques', 'overwhelmed', 'confusion', 'speeding', 'blackmailing'] 
	  } 
	   } 
	   } 
	 & \\ 
 

    \theutterance \stepcounter{utterance}  

    & \multicolumn{4}{p{0.6\linewidth}}{\cellcolor[rgb]{0.8,1,0.9}{%
	\makecell[{{p{\linewidth}}}]{% 
	  \tt {\tiny [A$\rangle$GM]}  
	 The first question is designed to determine if the target word is related to a job title.\\ \tt \\ \tt 1. QUESTION: Is the target word a job title?\\ \tt \\ \tt If yes, the possible words are postman or policeman. If no, consider actions (speeding, blackmailing), states (overwhelmed, confusion), or 'count.'\\ \tt \\ \tt 2. Next question would be to determine if it's specifically about delivering mail.\\ \tt \\ \tt 2. QUESTION: Does the target word relate to mail delivery?\\ \tt \\ \tt - Yes → postman\\ \tt - No → policeman\\ \tt \\ \tt If the answer is no, proceed to eliminate actions and focus on the remaining words.\\ \tt \\ \tt 3. If the second question was yes, then the target is 'postman.'\\ \tt \\ \tt 4. If the second question was no, then consider if it's a specific action or abstract concept.\\ \tt \\ \tt 5. Next, perhaps ask if the word is about an abstract emotion.\\ \tt \\ \tt 5. QUESTION: Is the target word related to an emotional state?\\ \tt \\ \tt - Yes → overwhelmed, confusion\\ \tt - No → speeding, blackmailing\\ \tt \\ \tt 6. If the answer is yes, choose between overwhelmed and confusion based on context clues.\\ \tt \\ \tt 7. If no, then it's either speeding or blackmailing. A guess can be made based on commonality or further clues if possible.\\ \tt \\ \tt After these steps, I would have a narrowed-down list to make an informed guess.\\ \tt \\ \tt \\ \tt GUESS: postman 
	  } 
	   } 
	   } 
	 & & \\ 
 

    \theutterance \stepcounter{utterance}  

    & & & \multicolumn{2}{p{0.3\linewidth}}{\cellcolor[rgb]{0.95,0.95,0.95}{%
	\makecell[{{p{\linewidth}}}]{% 
	  \tt {\tiny [GM$|$GM]}  
	 Invalid format. Guesser must use the form 'QUESTION: ' or 'GUESS: '. 
	  } 
	   } 
	   } 
	 & & \\ 
 

    \theutterance \stepcounter{utterance}  

    & & & \multicolumn{2}{p{0.3\linewidth}}{\cellcolor[rgb]{0.95,0.95,0.95}{%
	\makecell[{{p{\linewidth}}}]{% 
	  \tt {\tiny [GM$|$GM]}  
	 abort game 
	  } 
	   } 
	   } 
	 & & \\ 
 

\end{supertabular}
}

\end{document}
