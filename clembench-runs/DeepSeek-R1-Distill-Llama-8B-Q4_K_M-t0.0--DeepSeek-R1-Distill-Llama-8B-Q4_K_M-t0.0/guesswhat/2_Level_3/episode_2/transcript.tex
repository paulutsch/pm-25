
\documentclass{article}
\usepackage{colortbl}
\usepackage{makecell}
\usepackage{multirow}
\usepackage{supertabular}

\begin{document}

\newcounter{utterance}

\twocolumn

{ \footnotesize  \setcounter{utterance}{1}
\setlength{\tabcolsep}{0pt}
\begin{supertabular}{c@{$\;$}|p{.15\linewidth}@{}p{.15\linewidth}p{.15\linewidth}p{.15\linewidth}p{.15\linewidth}p{.15\linewidth}}

    \# & $\;$A & \multicolumn{4}{c}{Game Master} & $\;\:$B\\
    \hline 

    \theutterance \stepcounter{utterance}  

    & & \multicolumn{4}{p{0.6\linewidth}}{\cellcolor[rgb]{0.9,0.9,0.9}{%
	\makecell[{{p{\linewidth}}}]{% 
	  \tt {\tiny [A$\langle$GM]}  
	 Let's play a guessing game. Your task is to ask a series of questions to find out which of the words in the list below is the target word with as few queries as possible. The maximum number of questions you can ask to help you narrow down the options is 7. Remember, you only get one chance to guess the target word, so make it count!\\ \tt \\ \tt The rules of the game are: \\ \tt \\ \tt 1. Only ask one question per turn. DO NOT ASK MORE THAN ONE QUESTION.\\ \tt \\ \tt 2. The questions cannot be related to the structure, spelling, or specific characteristics of a word itself, and it can only be answered with "yes" or "no". \\ \tt \\ \tt 3. Use the following format for asking questions and DO NOT ADD ANY TEXT OTHER THAN THIS: \\ \tt \\ \tt QUESTION: <your question>\\ \tt \\ \tt 4. You have only one opportunity to guess the target word. To make your guess, use the following format and DO NOT ADD ANY TEXT OTHER THAN THIS: \\ \tt \\ \tt GUESS: <target word>\\ \tt \\ \tt Remember you cannot ask for a hint. Make your question brief. Let’s start!\\ \tt \\ \tt List of candidate words: ['Pear', 'Apple', 'Plum', 'Peach', 'Lemon', 'Orange', 'Pine', 'Cedar'] 
	  } 
	   } 
	   } 
	 & \\ 
 

    \theutterance \stepcounter{utterance}  

    & \multicolumn{4}{p{0.6\linewidth}}{\cellcolor[rgb]{0.8,1,0.9}{%
	\makecell[{{p{\linewidth}}}]{% 
	  \tt {\tiny [A$\rangle$GM]}  
	 The process involves systematically narrowing down the list by asking yes/no questions that split the remaining options in half each time. Here's how it would go step-by-step:\\ \tt \\ \tt 1. **Question 1:** Is the target word among "Cedar" or "Pine"?\\ \tt    - If yes: Proceed to narrow between these two.\\ \tt    - If no: Continue with other fruit trees.\\ \tt \\ \tt 2. **If Q1 is "Yes":**\\ \tt    - **Question 2:** Does the target contain the letter 'a'?\\ \tt      - If yes: It must be "Cedar".\\ \tt      - If no: It must be "Pine".\\ \tt \\ \tt 3. **If Q1 is "No":** The target is among "Apple", "Lemon", "Orange", "Peach", "Plum", or "Pine".\\ \tt    \\ \tt 4. **Question 2:** Does the target contain the letter 'm'?\\ \tt    - If yes: Target is either "Lemon" or "Plum".\\ \tt      - **Question 3:** Does it end with "on"? \\ \tt        - If yes: It's "Lemon".\\ \tt        - If no: It's "Plum".\\ \tt    - If no: Proceed to narrow among "Apple", "Orange", "Peach", and "Pine".\\ \tt \\ \tt 5. **If Q2 is "No":** Target is among "Apple", "Orange", "Peach", or "Pine".\\ \tt    \\ \tt 6. **Question 3:** Is the target among "Apple" and "Orange"? \\ \tt    - If yes: Further split.\\ \tt    - If no: Target is either "Peach" or "Pine".\\ \tt \\ \tt 7. **If Q3 is "Yes":** Ask if it starts with 'O' for "Orange" or else "Apple".\\ \tt    \\ \tt 8. **If Q3 is "No":** Determine between "Peach" and "Pine".\\ \tt \\ \tt After these steps, the target can be identified within 7 questions.\\ \tt \\ \tt GUESS: Lemon 
	  } 
	   } 
	   } 
	 & & \\ 
 

    \theutterance \stepcounter{utterance}  

    & & & \multicolumn{2}{p{0.3\linewidth}}{\cellcolor[rgb]{0.95,0.95,0.95}{%
	\makecell[{{p{\linewidth}}}]{% 
	  \tt {\tiny [GM$|$GM]}  
	 Invalid format. Guesser must use the form 'QUESTION: ' or 'GUESS: '. 
	  } 
	   } 
	   } 
	 & & \\ 
 

    \theutterance \stepcounter{utterance}  

    & & & \multicolumn{2}{p{0.3\linewidth}}{\cellcolor[rgb]{0.95,0.95,0.95}{%
	\makecell[{{p{\linewidth}}}]{% 
	  \tt {\tiny [GM$|$GM]}  
	 abort game 
	  } 
	   } 
	   } 
	 & & \\ 
 

\end{supertabular}
}

\end{document}
